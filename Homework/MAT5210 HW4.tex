\documentclass{article}
\usepackage{amsmath}
\usepackage{amssymb}

\title{MAT 5210 Homework 4}
\author{}
\date{}

\begin{document}

\maketitle

\begin{enumerate}
    \item Suppose $\beta$ is a root of an irreducible polynomial $f(X) = X^3 + pX + q \in \mathbf{Z}[X]$. Verify that $1, \beta, \beta^2, \beta^3$ have traces $3, 0, -2p, -3q$ respectively. Compute $\operatorname{tr}(\beta^4)$, and deduce that $\Delta(1, \beta, \beta^2)^2 = -4p^3 - 27q^2$.

    \item Suppose $\alpha$ is a root of a monic irreducible polynomial $f(X) \in \mathbf{Z}[X]$. Prove that if $\operatorname{deg}(f) = n$ and $K = \mathbf{Q}(\alpha)$, then show that
    \[
    \Delta(1, \alpha, \ldots, \alpha^{n-1})^2 = (-1)^{\frac{n(n-1)}{2}} \operatorname{Norm}_{K / \mathbf{Q}}(f'(\alpha))
    \]

    \item Let $K:\mathbf{Q}$ be a number field, and $\omega = \{\omega_1, \dots, \omega_n\}$ be a basis of $K$ (as a $\mathbf{Q}$-vector space). Let $\theta = \{\theta_1, \dots, \theta_n\}$ be a subset of $K$ such that
    $$\theta_j = \sum_{k=1}^n c_{kj} \omega_k \quad \quad (c_{kj} \in \mathbf{Q})$$
    for $1 \leq j \leq n$. 
    \begin{enumerate}
        \item[(a)] Prove that $\Delta(\theta) = \det(c_{ij})\Delta(\omega)$.
        \item[(b)] Hence show that $\theta$ is a basis of $K$ if and only if $\Delta(\theta) \neq 0$ (Hint: take $\omega = \{1, \alpha, \dots, \alpha^{n-1}\}$ in Question 2).
    \end{enumerate}
    
    \item Let $K$ be a number field. Prove that the ring of integers $\mathcal{O}_K$ is a Euclidean domain with Euclidean function
    \[
    \operatorname{Norm}_{K / \mathbf{Q}}: \mathcal{O}_K \setminus \{0\} \rightarrow \mathbf{N} \setminus \{0\}
    \]
    if and only if for all $\alpha \in K$, there exists $\beta \in \mathcal{O}_K$ such that $\left|\operatorname{Norm}_{K / \mathbf{Q}}(\alpha - \beta)\right| < 1$. Hence show that if $K = \mathbf{Q}(\sqrt{-d})$ with $d = 1, 2, 3, 7$, then $\mathcal{O}_K$ is a Euclidean domain. (Hint: consider the nearest point on a lattice).

    \item Suppose that $[K: \mathbf{Q}] = n$. We say an embedding $\sigma: K \hookrightarrow \mathbf{C}$ is real if $\sigma(K) \subset \mathbf{R}$, and complex if $\sigma(K) \subset \mathbf{C} \setminus \mathbf{R}$. Let there be a total of $r$ real embeddings, and $s$ pairs of conjugate complex embeddings $K \hookrightarrow \mathbf{C}$, with $n = r + 2s$. Show that if $\omega = \{\omega_1, \ldots, \omega_n\}$ is an integral basis of the ring of integers $\mathcal{O}_K$, then the sign of $\Delta(\omega)^2 \in \mathbf{Z}$ is $(-1)^s$. Verify this for $K = \mathbf{Q}(\alpha)$, where $\alpha^3 = m$ is a non-cube integer.

    \item
    \begin{enumerate}
        \item Show that $f(X) = X^3 - X + 2$ is an irreducible polynomial in $\mathbf{Z}[X]$.
        \item Let $\theta$ be a root of $f(X) = 0$. Calculate $\Delta(1, \theta, \theta^2)^2$.
        \item {\bf Stickelberger's Theorem} (see optional question) says that for any integral basis $\omega$ of $\mathcal{O}_K$, the discriminant $\Delta(\omega)^2$ satisfies $\Delta(\omega)^2 \equiv 0, 1 (\text{mod}\ 4)$. Use it to show that $\{1, \theta, \theta^2\}$ is an integral basis of $\mathcal{O}_K$.
    \end{enumerate}

    \item Suppose the monic polynomial $f \in \mathbf{Z}[X]$ satisfies Eisenstein's criterion at the prime $p$. Let $\alpha$ be a root of $f$ and $K = \mathbf{Q}(\alpha)$. Show that $\mathbf{Z}[\alpha]$ is a subgroup of $\mathcal{O}_K$, and the index
    \[
    [\mathcal{O}_K : \mathbf{Z}[\alpha]]
    \]
    is \underline{NOT} divisible by $p$.
\end{enumerate}

\noindent {\bf Optional Question:}
\begin{enumerate}
    \item (Stickelberger's Theorem) Let \( K \) be a number field with \( \sigma_i:  K \hookrightarrow \mathbf{C} \) being the embeddings of $\mathbf{C}$. 
    
    Suppose \( \omega = \{ \omega_1, \ldots, \omega_n \} \) is an integral basis of \( \mathcal{O}_K \), and \( \Delta(\omega) := \det(\sigma_i(w_j)) \). Let \( P \) be the sum of the positive terms of the expansion of the determinant \( \Delta(\omega) \), and let \( N \) be the sum of the negative terms of the determinant of \( \Delta(\omega) \), i.e. 
    $$\Delta(\omega) = \sum_{\tau \in A_n} \left(\prod_{i=1}^n \sigma_i(\omega_{\tau(i)})\right) - \sum_{\tau \notin A_n} \left(\prod_{i=1}^n \sigma_i(\omega_{\tau(i)})\right) = P - N.$$

\begin{itemize}
    \item[(a)] Let \( (\mathbf{C} \supseteq) M \supseteq K \supseteq Q \) be the `Galois closure' of $K:\mathbf{Q}$. Show that
    $$P+N,\ PN \in \mathbf{Q}\ (= M^{\mathrm{Gal}(M/\mathbf{Q})})$$
    \item[(b)] Consequently, show that
    $$\Delta(\omega)^2 = (P-N)^2 \equiv 0 \text{ or } 1\ (\mathrm{mod}\ 4).$$
\end{itemize} 



\end{enumerate}


\end{document}