\documentclass{article}
\usepackage{amsmath}
\usepackage{amsfonts}

\title{MAT 5210 Homework 1}
\author{}
\date{}

\begin{document}

\maketitle
\begin{enumerate}
    \item Let $G$ be a finite group acting on a finite set $X$, i.e. there is a group homomorphism $\sigma: G \to \mathrm{Aut}(X)$. We write $g \cdot x := (\sigma(g))(x)$.

\begin{enumerate}
    \item[(a)] Let $x \in X$. Show that the stabilizer of $x$,
    \[
    Stab_G(x) = \{ g \in G : g\cdot x = x \}
    \]
    is a subgroup of $G$.

    \item[(b)] \textbf{Orbit-Stabilizer Theorem:} For any $x \in X$, let 
    $$\mathrm{orb}(x) = \{ g\cdot x : g \in G \} \subseteq X$$ 
    be the orbit of $x$ under the action given by $G$. Show that $$|\mathrm{orb}(x)| = |G| / |Stab_G(x)|$$ 
    by showing that there is a bijection between the cosets of $Stab_G(x)$ and the elements of $\mathrm{orb}(x)$.

    \item[(c)] Show that if $\mathrm{orb}(x) \neq \mathrm{orb}(y)$, then $\mathrm{orb}(x) \cap \mathrm{orb}(y) = \emptyset$ and therefore that there exists a subset $Y$ of $X$ such that $X = \bigsqcup_{y \in Y} \mathrm{orb}(y)$.

    \item[(d)] Suppose that $G$ acts transitively on $X$ (i.e., for any $x \in X$, $\mathrm{orb}(x) = X$). In addition, suppose that $|X| > 1$. Show that there exists $g \in G$ such that $g \cdot x \neq x$ for any $x \in X$.

    \item[(e)] Let $g \in G$. Define a map $\psi_g : G \to G$ as follows: for any $h \in G$, $\psi_g(h) = ghg^{-1}$. Show that $\psi_g$ is an automorphism and that $g \mapsto \psi_g$ is a homomorphism of $G$ into $\operatorname{Aut}(G)$. Let $H = \{ \psi_g \mid g \in G \}$ (one usually refers to $H$ as the group of inner automorphisms of $G$). Show that $H$ is a group and that it is normal in $\operatorname{Aut}(G)$.

    \item[(f)] Let $Z(G) = \{ g \in G : ghg^{-1} = h \ \forall h \in G \}$ be the center of $G$. Show that $Z(G)$ is normal. In addition, show that if $G / Z(G)$ is cyclic, then $G$ is abelian.

    \item[(g)] Using (e) and (f) (or otherwise), show that if $\operatorname{Aut}(G)$ is cyclic, then $G$ is abelian.
\end{enumerate}

\item Find the minimal polynomial for $\displaystyle \frac{\sqrt{3}}{1 + 2^{1/3}}$ over $\mathbb{Q}$; that is, the monic polynomial $m(x) \in \mathbb{Q}[x]$ of smallest possible degree satisfying
\[
m\left( \frac{\sqrt{3}}{1 + 2^{1/3}} \right) = 0.
\]

\item Show that if $a \in \mathbb{Z}$ is divisible by a prime $p$ but not by $p^2$, then $x^n - a$ is irreducible over $\mathbb{Q}$ for all $n \geq 1$. Show also that it has no repeated roots in any extension of $\mathbb{Q}$.

\item Recall the formal derivative $D: K[x] \to K[x]$ is defined by
\[
D\left( a_0 + a_1 x + \cdots + a_n x^n \right) = a_1 + 2a_2 x + \cdots + n a_n x^{n-1}.
\]
Prove that if $a, b \in K$ and $f, g \in K[x]$, then
\begin{enumerate}
    \item[(a)] $D(af + bg) = aDf + bDg$;
    \item[(b)] $D(fg) = fDg + gDf$;
    \item[(c)] $Dh(x) = Dg(x)Df(g(x))$ when $h(x) = f(g(x))$.
\end{enumerate}
If $a \in K$, show that
\begin{enumerate}
    \item[(d)] $(x - a)$ divides $f(x)$ in $K[x]$ if and only if $f(a) = 0$;
    \item[(e)] $(x - a)^2$ divides $f(x)$ in $K[x]$ if and only if $f(a) = 0 = Df(a)$.
\end{enumerate}
Deduce that if the polynomials $f$ and $Df$ are relatively prime in $K[x]$, then $f$ has no multiple root.

\item 
\begin{enumerate}
    \item[(a)] Show that if $m$ is any positive integer, then the polynomial $x^{p^m} - x$ has no repeated root in any extension of fields $L : \mathbb{F}_p$. 
    \item[(b)] Let
\[
K = \{ \alpha \in L : \alpha^{p^m} = \alpha \}
\]
be the set of roots of $x^{p^m} - x$ in the extension $L$. Show that $K$ is a subfield of $L$.
\item[(c)] Let $n$ be a positive integer. Show that if $m$ divides $n$, then $p^m - 1$ divides $p^n - 1$ in $\mathbb{Z}$ and $x^{p^m} - x$ divides $x^{p^n} - x$ in $\mathbb{F}_p[x]$.
\end{enumerate}

\item Let $E : F$ be an extension field of prime degree $\ell$, and let $\alpha \in E \setminus F$. Let $M_\alpha$ be the $F$-linear map induced by the multiplication by $\alpha$:
\[
\begin{aligned}
M_\alpha : E &\longrightarrow E \\
u &\mapsto \alpha \cdot u
\end{aligned}
\]
Show that the characteristic polynomial of $M_\alpha$ is equal to the minimal polynomial of $\alpha$. \textbf{Hint:} Cayley-Hamilton.

\item
\begin{enumerate}
    \item[(a)] Let $f(x) = x^3 - s_1 x^2 + s_2 x - s_3 = (x - \alpha)(x - \beta)(x - \gamma) \in \mathbb{Q}[x]$ where $\alpha, \beta, \gamma \in \mathbb{C}$. Denoting $\sigma_i = \alpha^i + \beta^i + \gamma^i$ for $i \geq 0$, show that $\sigma_0 = 3$, $\sigma_1 = s_1$, and $\sigma_2 = s_1^2 - 2s_2$. Show further that
    \[
    \sigma_r = s_1 \sigma_{r-1} - s_2 \sigma_{r-2} + s_3 \sigma_{r-3}
    \]
    for all $r \geq 3$.

    \item[(b)] Let $\delta = (\alpha - \beta)(\alpha - \gamma)(\beta - \gamma)$ and $\Delta = \delta^2$. Show that
    \[
    \Delta = -4s_1^3 s_3 + s_1^2 s_2^2 + 18s_1 s_2 s_3 - 4s_2^3 - 27s_3^2.
    \]
    \textbf{Hint:} You may find it useful to consider the Vandermonde determinant
    \[
    \operatorname{det} \begin{pmatrix}
    1 & 1 & 1 \\
    \alpha & \beta & \gamma \\
    \alpha^2 & \beta^2 & \gamma^2
    \end{pmatrix}
    \]
    and the determinant of this matrix multiplied by its transpose to deduce first that
    \[
    \Delta = \operatorname{det} \begin{pmatrix}
    \sigma_0 & \sigma_1 & \sigma_2 \\
    \sigma_1 & \sigma_2 & \sigma_3 \\
    \sigma_2 & \sigma_3 & \sigma_4
    \end{pmatrix}.
    \]
\end{enumerate}
\end{enumerate}
\end{document}